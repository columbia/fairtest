\section{Motivation and Theoretical Background}
\label{sect:motivation}

In this section we motivate our work and establish the theoretical
background necessary for building \sysname. We start with examples
that intuitively  introduce the notion of discriminatory treatment
of users of modern data-driven applications. Then, we formalize 
{\em statistical parity} and {\em relaxed statistical parity} to
quantify the presence of discriminatory treatment of populations.

As a first, simple example consider a hypothetical  online store that
sells product in either ``low'' or ``hight'' prices. That is, each
users sees either a ``low'' or a ``high'' price upon visiting the web
site of hypothetical store. Also, suppose that this hypothetical online
store has 100 active users that are spread into three different groups
A, B, and C, as shown in Table~\ref{tab:DiscriminationExample}.
Intuitively, users of
the population A are treated unfavorably compared to users of populations
of B and C. This is because users who belong to population A receive high
prices more often than users who belong to the populations B and C. In other
words, the probability that a user will be presented with a high price is
higher if he or she belong to population A, and lower if he or she belongs to
population B and C. Therefore, there is no {\em statistical parity} among the
users of populations A, B, and C, since the hypothetical online store
customizes price and discriminates based on users' membership on a population.

On the other hand, for the same hypothetical online store, consider
that the users are distributed into three populations A, B, and C, as
shown in Table~\ref{tab:NonDiscriminationExample}. In this case
users of each population see approximately the same proportion of high versus
low prices. In other words, a user has approximately the same probability to
receive a high or a low price, regardless of the population to which he or she
belongs. Therefore, there is no population whose users are treated unfavorably.

The previous two motivating examples intuitively leads us use {\em statistical
parity} as a simple metric indicating discriminatory treatment of populations.
In what follows, we more formally introduce the condition asserting
{\em statistical parity} among users of populations.


\subsection{Statistical Parity}
\label{sect:statparity}

For two sets of users S and S', and some output O, {\em statistical parity}
asks that:
\begin{equation}
|P\{O | x \in S\} - P\{O | x \in T\}| \le \varepsilon
\label{eq:StatisticalParity}
\end{equation}
Note that if users are split into more than two populations then statistical
parity requires that condition~\ref{eq:StatisticalParity} holds for separately
for each population against all the rest.

Having introduced a formal metric for asserting {\em statistical parity}, we now
revisit Tables~\ref{tab:DiscriminationExample}
and~\ref{tab:NonDiscriminationExample}.
In Table~\ref{tab:DiscriminationExample}, we observe that for population A the
delta of condition~\ref{eq:StatisticalParity} is $0.198$ for high prices.
While, in Table~\ref{tab:NonDiscriminationExample}, for population A the delta
of condition~\ref{eq:StatisticalParity} is $0.028$ for high price. Observe that
the value of the delta in the first case is one order of magnitude higher than
the respective value in the second case. This is because in the first case
users of A are receiving a high price much more often than in the second case.
Hence, in the first case there is low {\em statistical parity} among users of
populations A, B, and C; while in the second case, there is high statistical
parity among users of populations A, B, and C.

\begin{table}[t]
{
 \scriptsize
  \renewcommand{\arraystretch}{1.5}
  \begin{tabular}{ c | c | c  c | c }
    & & \multicolumn{2}{|c|}{\underline{Price}} &  Statistical Parity\\
    Population & \#Members & Low & High & (for high price) \\
    \hline
    A & 30 &  10 & 20 & $0.198 = | \frac{20}{30} - \frac{33}{70}|$ \\
    B & 30 &  16 & 14 & $0.090$ \\
    C & 40 &  21 & 19 & $0.091$ \\
    \hline
    Total & \#100 & 47 & 53 & - \\
  \end{tabular}
  \caption{{\bf Discriminatory behavior against population(s).} Users of
    population A receive twice as many high prices as low, while users of
    populations B and C receive approximately the same amount of high and low
    prices. Therefore, the probability that a user will receive a high price
    depends on the population to which he or she belongs to, and
    condition~\ref{eq:StatisticalParity} for {\em statistical parity} yields a
    higher delta compared to the previous example.
  }
  \label{tab:DiscriminationExample}
}
\end{table}


Although simple and intuitive, the above definition of {\em statistical parity}
fails in cases where developers implement policies that inherently discriminate
users based on some requirement. Such cases may arise when there is a
business necessity which is relevant to the policy of pricing, advertising,
or hiring. For instance, consider the example of a hypothetical bank that issues
two types of loans: (a) payday loans that are issued to customers without credit
history and have high interest rate, and (b) personal loans that are issued to
customers with credit history and have low interest. Suppose that this
hypothetical bank serves 300 customers as shown in
Table~\ref{tab:BusinessNessecityA}.

At first sight users of population A are being treated unfavorably because they
are proportionally taking more payday loans than users of population B and C.
Specifically, users of population A
receive 25\% and 20\% more payday loans than users of populations B and C,
respectively. However, upon closer examination, one observes that only 20\% of A’s
users have credit history (which is a prerequisite for personal loans) against
55\% of B’s and C’s users. Therefore, business necessity requires that before
examining {\em statistical parity}, users should be discriminated based on whether
they have credit history or not. Since in presence of business necessity
satisfying {\em statistical parity} is not reasonable, in what follows we introduce
a relaxed version of {\em statistical parity}.

\begin{table}[t]
{\scriptsize
  \renewcommand{\arraystretch}{1.5}
  \begin{tabular}{ c | c | c  c | c }
    & & \multicolumn{2}{|c|}{\underline{Price}} &  Statistical Parity\\
    Population & \#Members & Low & High & (for high price) \\
    \hline
    A & 30 &  15 & 15 & $0.028 = | \frac{15}{30} - \frac{37}{70}|$ \\
    B & 30 &  14 & 16 & $0.019$ \\
    C & 40 &  19 & 21 & $0.008$ \\
    \hline
    Total & \#100 & 48 & 52 & - \\
  \end{tabular}
  \caption{{\bf Non-discriminatory behavior.} Users of three populations receive approximately
  the same proportions of low versus high prices. Therefore, the probability that a user
  will receive a high price is independent of the population to which he or she belongs to,
  and condition~\ref{eq:StatisticalParity} for statistical parity yields a low delta.}
  \label{tab:NonDiscriminationExample}
}
\end{table}



\subsection{Relaxing Statistical Parity}
\label{sect:relaxedstatparity}
\begin{table*}[t]
{ \small
  \center
  \renewcommand{\arraystretch}{1.5}
  \begin{tabular}{ c | c c c | c c c | c c c}
    Credit
    & \multicolumn{3}{|c|}{\underline{Loan Type (Population A)}}
    & \multicolumn{3}{|c}{\underline{Loan Type (Population B) }}
    & \multicolumn{3}{|c}{\underline{Loan Type (Population C) }} \\
    history & Payday & Personal & Total & Payday & Personal & Total & Payday & Personal & Total \\
    \hline
    YES & 5  & 15 & 20 & 15 & 40 & 55 & 10 & 45 & 55 \\
    NO  & 80 & 0  & 80 & 45 & 0 & 45 & 45 & 0 & 45\\
    \hline
    Total & 85 & 15 & 100 & 60 & 40 & 100 & 55 & 45 & 100\\
  \end{tabular}
  \caption{{\bf Discriminatory behavior on presence of business necessity (credit history).}
    At first sight users of population A are proportionally taking more payday loans
    (payday loans come with higher interest than personal loans) than users of population B.
    Specifically, users of population A receive 25\% more and 20\% more  payday loans than
    users of populations B and C, respectively. However, upon closer examination, one notes
    that only 20\% of A's users have credit history (which is a prerequisite for personal
    loans) against 55\% of B's and C's users. Therefore, business necessity requires that
    before examining statistical parity, users should be discriminated based on whether
    they have credit history or not.
  }
  \label{tab:BusinessNessecityA}
}
\end{table*}

\begin{table*}[t]
{ \small
  \center
  \renewcommand{\arraystretch}{1.5}
  \begin{tabular}{ c | c c c | c c | c c}
    Credit
    & \multicolumn{3}{|c|}{\underline{Loan Type (Population A)}}
    & \multicolumn{2}{|c}{\underline{Statistical Parity (Population A) }}
    & \multicolumn{2}{|c}{\underline{Relaxed Statistical Parity (Population A) }} \\
    history & Payday & Personal & Total & Payday & Personal & Payday & Personal \\
    \hline
    YES & 5  & 15 & 20 & - & - &  0.022 & 0.022 \\
    NO  & 80 & 0  & 80 & - & - &  0    & 0 \\
    \hline
    Total & 85 & 15 & 100 & 0.275 & 0.70 & - & - \\
  \end{tabular}
  \caption{{\bf Relaxing statistical parity on presence of business necessity
    (credit history).}
    Without considering business necessity, i.e., credit history,
    condition~\ref{eq:StatisticalParity} for statistical parity yields a
    higher delta than if we
    consider business necessity, let the users be discriminated on whether
    they have credit
    history or not, and apply conditions~\ref{eq:RelaxedStatisticalParityA}
    and~\ref{eq:RelaxedStatisticalParityB}.
  }
  \label{tab:BusinessNessecityB}
}
\end{table*}
For a sets of users S and S', a requirement R (business necessity), and some
output O, {\em relaxed statistical parity} asks that:
\begin{equation}
|P\{O | x \in S \cap R\} - P\{O | x \in T \cap R\}| \le \varepsilon
\label{eq:RelaxedStatisticalParityA}
\end{equation}
and
\begin{equation}
|P\{O | x \in S \cap R'\} - P\{O | x \in T \cap R'\}| \le \varepsilon
\label{eq:RelaxedStatisticalParityB}
\end{equation}

Essentially, conditions~\ref{eq:RelaxedStatisticalParityA}
and~\ref{eq:RelaxedStatisticalParityB} let the users be discriminated into two
sets R (users that meet the business necessity) and R' (users that do not
meet the business necessity). However, conditioned on being in R (or not),
there should be no additional discrimination between the sets S and S.
This definition
could easily be extended to deal with non-binary categories of utility, where
business-necessity implies splitting the user base into multiple sets.

Having established the notion of {\em relaxed statistical parity} we now revisit
the example of a hypothetical bank that issues payday and personal loans to 300
users, and compare {\em statistical payday} against {\em relaxed statistical
parity}. The users of the hypothetical bank are again distributed into three
populations, as shown in Table~\ref{tab:BusinessNessecityA}. Also,
Table~\ref{tab:BusinessNessecityB} demonstrates the {\em statistical parity}
and the {\em relaxed statistical parity} for the users of population A.
The delta
of condition~\ref{eq:StatisticalParity} for {\em statistical parity} of
population A (middle column of Table~\ref{tab:BusinessNessecityB}) is $0.27$
and $0.70$ for payday and personal loans respectively. This implies
discriminatory (unfavorable) treatment of users of population A. Indeed, there
is such a discriminatory behavior that can be explained from an a-priory known
credit history requirement -- or, business necessity. Towards incorporating
business necessity, the right-hand side column of
Table~\ref{tab:BusinessNessecityB} presents the values for the delta of
conditions~\ref{eq:RelaxedStatisticalParityA}
and~\ref{eq:RelaxedStatisticalParityB}. In this case the deltas are $0.02$
(compared to $0.27$ and $0.70$), because users are split into two sets
based on whether they qualify for a personal loan or not, and then
{\em relaxed statistical parity} is examined separately.

Based on the theoretical background discussed in Section~\ref{sect:statparity}
and~\ref{sect:relaxedstatparity} we design and implement \sysname; a service
which, at its core, uncovers -- and reports as {\em privacy bugs} -- any violations
of statistical parity among users of different populations.


