\section{FairTest}
\label{sect:fairtest}

% What we built

\begin{figure}[h]
 \includegraphics[width=0.49\textwidth]
  {\detokenize{figures/overview}}
  \caption{ {\bf \textit{FairTest} Use Overview}. Gives an overview of how an application
    may use \sysname to generate bug-reports.
  }
  \label{fig:FairtestOverview}
\end{figure}

In this section we present the design and implementation of \sysname.
At its core, \sysname assets that there are no violations of statistical
condition~\ref{eq:StatisticalParity} of Section~\ref{sect:statparity}.
%To address the descrimination test, We designed \textit{FairTest}, a software
%testing framework for web-based applications to find out potential
%\textit{privacy bugs}, whose definition was just described. Its basic workflow
Figure~\ref{fig:FairtestOverview}, gives an overview of how a data-driven
application with a pricing policy ``P'' can use \sysname to generate
bug reports. As shown in Figure~\ref{fig:FairtestOverview}, user-attributes
(protected and non-protected) and price engine's output are used by
\sysname to generate bug-reports. \sysname generates these report based on
on the {\em statistical parity} of users on certain protected attributes
(such as gender, race, etc.). In specific, any violation of
condition~\ref{eq:StatisticalParity}, is reported by \sysname as a {\em privacy bug}.

% Target user / characteristic

\sysname is designed to be used for designers and engineers of 
data-based web applications. In addition, it can also be used for auditors 
or government agencies to look for evidence on certain parity violation,
such as in the Staples case. The framework, for generality, is interfaced 
with RESTful API that can easily connect to existing software solutions.
We expect that developers will only make minimal modifications to the
application (mostly adding the tracking code) to derive the data for audit
in \sysname. Also, to address privacy on the testing framework,
\sysname does not require identifiable user information, such as
exact address, name, etc. The framework itself is implemented in Python and
Django, with support to both SQLite or PostgreSQL database.

% Data got

Once \textit{FairTest} has information about the users and the corresponding
outputs from the software being tested, it can generate statistical reports
evaluating the statistical parity between the protected info, the actual
input and the output. This info can then be reported either through a RESTful
API or a web interface to the developer or tester. With the info, we can then
suggest potential \textit{privacy bugs} that are caused by the statiscal
parity found by the system, offering suggestion to the developer if the
customization engine appears acceptable or not.

% Subsections

This section will be divided into the following parts. In subsection 3.1, the
\textit{FairTest} archetecture will be discussed; and subsection 3.2 will
discuss the API design of \textit{FairTest}. 

\subsection{Architecture}

% Overview
Our \sysname implementation consists of three parts:
a RESTful input API that gathers data from monitored application; a
statistical engine that asserts statistical parity between protected
user attributes; and a reporting tool that generates bug-reports.
Figure~\ref{fig:FairtestArch} demonstrates this architecture.

\begin{figure}[h]
 \includegraphics[width=0.49\textwidth]
  {\detokenize{figures/architecture}}
  \caption{{\bf \textit{FairTest}'s Architecture}. The diagram shows \sysname's architecture,
    along with a very simple application using it. Bold lines represent API
    calls and dash lines represent bugreport queries. Solid line indicate data flow
    as part of the application or of the testing framework.}
  \label{fig:FairtestArch}
\end{figure}

\textit{FairTest} includes three parts: a RESTful input API that gathers data
from the application tested; a statistical engine that finds statistical parity
between sentitive user data and the price descrimination, and a reporting tool
showing the results.

In a sample workflow, the input API receives requests from the monitored
application, and records all outputs with regard to a user's characteristics. 
As it uses common web framwork (Django) and database, this API can, in fact, be
distributed across a set of servers that share a same database; this can
provide rate-limitation and fault-tolerance in case the system is deployed in
production and to a large scale. 

When, at a specific timeframe, the report needs to be shown, the data analyzer
can be run to apply the algorithm. The data analyzer currently runs as a single
procedure to be run on a server; but it can be designed to be scalable to
multiple machines with distributed computing frameworks. After the data is
analyzed, the \textit{privacy bugs} found will be shown to the user.

\subsection{API}

\textit{FairTest} incorporates a RESTful API for gethering data from the
appliation. The basic functions of it is included in Table~\ref{tab:fairtestApi}.

\begin{table}[t]
{
 \scriptsize
  \renewcommand{\arraystretch}{1.5}
  \begin{tabular}{ l | l }
    Usage & Function \\
    \hline
    POST /user & Add a user with attributes \\
    POST /output & Add a output from the program \\
    POST /user/12/output & Link an output to a user\\
    \hline
    PUT /user/12 & Change attributes of user \#12 \\
    \hline
    GET /user/12 & Get information of user \#12 \\
    GET /user/12/output & Get outputs for user \#12 \\
  \end{tabular}
  \caption{\bf Data input API for FairTest} 
  \label{tab:fairtestApi}
}
\end{table}

As can be seen, this API contains very simple operations to be linked into
the application. As a RESTful API, it also has support in most programming
languages to add the outputs directly from the MVC models. It also offers
a simple querying mechanism for the developer to check validity of data.

As the data model is simple, we expect this API to be scalable as well.
