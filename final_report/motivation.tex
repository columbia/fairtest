%  \item Introduce the notion of Fairness in a data driven world. Refer again to the case mentioned
%    in paragraph two (Staples case in detail). Then give specific examples with numbers and tables
%    to simulate discrimination an underline lack (or not) of fairness.
%  \item Analyze the tables and introduce the notion of statistical parity.
%  \item Foreshadow the notion of business necessity.
%  \item Introduce an example with business necessity (requirement).
%  \item Analyze the example based only on statistical parity.
%  \item Contradict and introduce business necessity.
%  \item Adreess that we limit our scope into statistical parity wnhen we build Fairtest

\section{Motivation and Theoretical Background}
In this section we present a set of motivating examples that help
(i) introduce the notion of discriminatory treatment of users of modern
data-driven applications, and (ii) quantify the presence of
discriminatory treatment of seemingly unrelated users.

As a first, very simple example consider a web application which is
the front end of an online store. Suppose that this hypothetical online
store has 100 active users that are spread into three different groups
A, B, and C, as shown in Table~\ref{tab:DiscriminationExample}. Also,
suppose that upon visiting, a user sees either a low or high price, as
shown in Table~\ref{tab:DiscriminationExample}. Intuitivelly, users of
the population A are treated unfavorably compared to users of populations
of B and C, since users of the former population receive high prices
more often than users of the later population. In other words, the
probability that a user will be presented with a high price is higher
if he or she belong to population A, and lower if he or she belongs to
population B and C. Therefore, there is no statistical parity among the
users of populations A, B, and C, since the hypothetical online store
optimizes its price and disciminates based on users' membership on a
population.

On the other hand, for the same hypothetical online store, consider
that the users are distributed into three populations A, B, and C, as
shown in Table~\ref{tab:NonDiscriminationExample}. In this case it is
apparent that a user has approximately the same probability to receive
a high or a low price, regardless of the population to which he or she
belongs. In other words, there is no population whose users are treated
infavorably.

The previous two motivating examples leads us to the following definition
of statistical parity, in order to quantify the notion of discriminatory
treatment of populations.

\subsection{Statistical Parity}
For two sets of users S and S', and some output O, statistical parity
asks that:
\begin{equation}
|P\{O | x \in S\} - P\{O | x \in T\}| \le \varepsilon
\label{eq:StatisticalParity}
\end{equation}
Note that if users are split into more than two populations then statistical
parity requires that condition~\ref{eq:StatisticalParity} holds for seperately
for each population against all the rest.

Having introduced a formal metric
for statistical parity, we now revisit Tables~\ref{tab:DiscriminationExample}
and~\ref{tab:NonDiscriminationExample}. In Table~\ref{tab:DiscriminationExample},
we observe that for population A the delta of condition~\ref{eq:StatisticalParity}
is $0.198$ for high price. While, in Table~\ref{tab:NonDiscriminationExample}, for
population A the delta of condition~\ref{eq:StatisticalParity} is $0.028$ for
high price. The value of the delta in the first case is one order of magnitude
higher than in the second case, because users of A are receiving a high
price much more often and therefore there is very low statistical parity among
users of populations A, B, and C. On the other hand, in the second case a
user is equially probable to receive either a low or a high price, regardless
of the population to which he or she belongs, and therefore there is a high
statistical parity among users of populations A, B, and C.

\begin{table}[t]
{
 \scriptsize
  \renewcommand{\arraystretch}{1.5}
  \begin{tabular}{ c | c | c  c | c }
    & & \multicolumn{2}{|c|}{\underline{Price}} &  Statistical Parity\\
    Population & \#Members & Low & High & (for high price) \\
    \hline
    A & 30 &  10 & 20 & $0.198 = | \frac{20}{30} - \frac{33}{70}|$ \\
    B & 30 &  16 & 14 & $0.090$ \\
    C & 40 &  21 & 19 & $0.091$ \\
    \hline
    Total & \#100 & 47 & 53 & - \\
  \end{tabular}
  \caption{{\bf Discriminatory behavior against population(s).} Users of population A receive twice
  as many high prices as low, while users of populations B and C receive approximately the same
  ammount of high and low prices. Therefore, the probability that a user will receive a high price
  depends on the population to which he or she belongs, and condition ~\ref{eq:StatisticalParity}
  for statistical parity yields a higher delta compared to the previous example. }
  \label{tab:DiscriminationExample}
}
\end{table}

Although simple and intuitive, the above definition of statistical parity fails
in cases where users are inherently discrimintated based on an attribute that
constitutes a business necessity for a policy implemented by the developer.
To be more specific, consider the example of a hypothetial bank that issues two
types of loans: (a) payday loans that are issued to customers without credit
history and have high interest rate, and (b) personal loans that are issued to
customers with credit history and have low interest. Suppose that this
hypothetical bank serves 300 customers as shown in
Table~\ref{tab:BusinessNessecity}.



\subsection{Relaxing Statistical Parity}
There may be situations in which satisfying statistical parity is not reasonable. Indeed, disparate
impact is in itself not necessarily illegal, if the discriminatory practice can be justified
through some notion of business necessity. Although the notion of business necessity was
originally considered for discrimination in hiring, it can reasonably be applied to other situations
of interest such as ad targeting, mortgage or (life) insurance

In many examples where business necessity is brought up, business utility is defined very
simply by the presence of some particular ability or attribute. For instance, a truck
driving company would argue that possession of a truck drivers license is a business necessity,
which would justify that their hiring process is discriminatory against women.
This leads to a simple binary definition of business necessity. Let R denote the set of
users that satisfy the business requirements. Then, a relaxed form of statistical parity would
require that

Essentially, we let the users be discriminated into R and Rc (this is business necessity).
However, we want to make sure that conditioned on being in R (or not), there is no additional
discrimination between sets S and T .
This definition could easily be extended to deal with non-binary categories of utility,
where business-necessity implies splitting the user base into multiple sets

For a sets of users S and T and some output O, relaxed statistical parity asks that:
\begin{equation}
|P\{O | x \in S \cap R\} - P\{O | x \in T \cap R\}| \le \varepsilon
\label{eq:RelaxedStatisticalParityA}
\end{equation}
and
\begin{equation}
|P\{O | x \in S \cap R'\} - P\{O | x \in T \cap R'\}| \le \varepsilon
\label{eq:RelaxedStatisticalParityB}
\end{equation}


\begin{table}[t]
{\scriptsize
  \renewcommand{\arraystretch}{1.5}
  \begin{tabular}{ c | c | c  c | c }
    & & \multicolumn{2}{|c|}{\underline{Price}} &  Statistical Parity\\
    Population & \#Members & Low & High & (for high price) \\
    \hline
    A & 30 &  15 & 15 & $0.028 = | \frac{15}{30} - \frac{37}{70}|$ \\
    B & 30 &  14 & 16 & $0.019$ \\
    C & 40 &  19 & 21 & $0.008$ \\
    \hline
    Total & \#100 & 48 & 52 & - \\
  \end{tabular}
  \caption{{\bf Non-discriminatory behavior.} Users of three populations receive approximately
  the same proportions of low versus high prices. Therefore, the probability that a user
  will receive a high price is independent of the population to which he or she belongs,
  and condition~\ref{eq:StatisticalParity} for statistical parity yields a low delta.}
  \label{tab:NonDiscriminationExample}
}
\end{table}




\begin{table*}[t]
{ \small
  \center
  \renewcommand{\arraystretch}{1.5}
  \begin{tabular}{ c | c c c | c c | c c}
    Credit
    & \multicolumn{3}{|c|}{\underline{Loan Type (Population A)}}
    & \multicolumn{2}{|c}{\underline{Statistical Parity (Population A) }}
    & \multicolumn{2}{|c}{\underline{Relaxed Statistical Parity (Population A) }} \\
    history & Payday & Personal & Total & Payday & Personal & Payday & Personal \\
    \hline
    YES & 5  & 15 & 20 & - & - &  0.022 & 0.022 \\
    NO  & 80 & 0  & 80 & - & - &  0    & 0 \\
    \hline
    Total & 85 & 15 & 100 & 0.275 & 0.70 & - & - \\
  \end{tabular}
  \label{tab:BusinessNessecityA}
  \caption{{\bf Relaxing statistical parity on presence of bussiness necessity (credit history).}
    Without considering business necessity, i.e., credit history,
    condition~\ref{eq:StatisticalParity} for statistical parity yields a higher delta than if we
    consider business necessity, let the users be discriminated on whether they have credit
    history or not, and apply conditions~\ref{eq:RelaxedStatisticalParityA}
    and~\ref{eq:RelaxedStatisticalParityB}.}
} \end{table*}



\begin{table*}[t]
{ \small
  \center
  \renewcommand{\arraystretch}{1.5}
  \begin{tabular}{ c | c c c | c c c | c c c}
    Credit
    & \multicolumn{3}{|c|}{\underline{Loan Type (Population A)}}
    & \multicolumn{3}{|c}{\underline{Loan Type (Population B) }}
    & \multicolumn{3}{|c}{\underline{Loan Type (Population C) }} \\
    history & Payday & Personal & Total & Payday & Personal & Total & Payday & Personal & Total \\
    \hline
    YES & 5  & 15 & 20 & 15 & 40 & 55 & 10 & 45 & 55 \\
    NO  & 80 & 0  & 80 & 45 & 0 & 45 & 45 & 0 & 45\\
    \hline
    Total & 85 & 15 & 100 & 60 & 40 & 100 & 55 & 45 & 100\\
  \end{tabular}
  \label{tab:BusinessNessecity}
  \caption{{\bf Discriminatory behavior on presence of bussiness necessity (credit history).}
  At first sight users of population A are proportionaly taking more payday loans
  (payday loans come with higher interest than personal loans) than users of population B.
  Specifically, users of population A receive 25\% more and 20\% more  payday loans than
  users of populations B and C, respectively. However, upon closer examination, one notes
  that only 20\% of A's users have credit history (which is a prerequisite for personal
  loans) against 55\% of B's and C's users. Therefore, bussiness necessity requires that
  before examining statistical parity, users should be discriminated based on whether
  they have credit history or not.}
}
\end{table*}
