\section{Motivation}
\begin{itemize}
  \item Introduce the notion of Fairness in a data driven world. Refer again to the case mentioned
    in paragraph two (Staples case in detail). Then give specific examples with numbers and tables
    to simulate discrimination an underline lack (or not) of fairness.
  \item Analyze the tables and introduce the notion of statistical parity.
  \item Foreshadow the notion of business necessity.
  \item Introduce an example with business necessity (requirement).
  \item Analyze the example based only on statistical parity.
  \item Contradict and introduce business necessity.
  \item In addition to statistical parity, analyze also considering
    business necessity.
  \item Foreshadow natural inclination/utility function.
  \item Adreess that we limit our scope without adressing system utility; we leave it for future work.
\end{itemize}



\begin{table}[!h]
{
  \renewcommand{\arraystretch}{1.5}
  \begin{tabular}{ c | c | c  c | c }
    & & \multicolumn{2}{|c|}{\underline{Price}} &  Statistical Parity\\
    Population & \#Members & Low & High & (for high price) \\
    \hline
    A & 30 &  15 & 15 & $0.028 = | \frac{15}{30} - \frac{37}{70}|$ \\
    B & 30 &  14 & 16 & $0.019$ \\
    C & 40 &  19 & 21 & $0.008$ \\
    \hline
    Total & \#100 & 48 & 52 & - \\
  \end{tabular}
  \caption{{\bf Non-discriminatory behavior.} Users of three populations receive approximately
  the same proportions of low versus high prices. Therefore, the probability that a user
  will receive a high price is independent of the population to which he or she belongs,
  and the condition for statistical parity yields a low delta.}
  \label{tab:S}
} \end{table}

\begin{table}[!h]
{
  \renewcommand{\arraystretch}{1.5}
  \begin{tabular}{ c | c | c  c | c }
    & & \multicolumn{2}{|c|}{\underline{Price}} &  Statistical Parity\\
    Population & \#Members & Low & High & (for high price) \\
    \hline
    A & 30 &  10 & 20 & $0.198 = | \frac{20}{30} - \frac{33}{70}|$ \\
    B & 30 &  16 & 14 & $0.090$ \\
    C & 40 &  21 & 19 & $0.091$ \\
    \hline
    Total & \#100 & 47 & 53 & - \\
  \end{tabular}
  \caption{{\bf Discriminatory behavior against population(s).} Users of population A receive twice
  as many high prices as low, while users of populations B and C receive approximately the same
  ammount of high and low prices. Therefore, the probability that a user will receive a high price
  depends on the population to which he or she belongs, and the condition for statistical parity
  yields a higher delta compared to the previous example. }
  \label{tab:S}
} \end{table}

\begin{table*}[t]
{\center
  \renewcommand{\arraystretch}{1.5}
  \begin{tabular}{ c | c c c | c c c | c c c}
    Credit
    & \multicolumn{3}{|c|}{\underline{Loan Type (Population A)}}
    & \multicolumn{3}{|c}{\underline{Loan Type (Population B) }}
    & \multicolumn{3}{|c}{\underline{Loan Type (Population C) }} \\
    history & Payday & Personal & Total & Payday & Personal & Total & Payday & Personal & Total \\
    \hline
    YES & 5  & 15 & 20 & 15 & 40 & 55 & 10 & 45 & 55 \\
    NO  & 80 & 0  & 80 & 45 & 0 & 45 & 45 & 0 & 45\\
    \hline
    - & 85 & 15 & 100 & 60 & 40 & 100 & 55 & 45 & 100\\
  \end{tabular}
  \caption{{\bf Discriminatory behavior and bussiness necessity.} At first sight users of
  population A are proportionaly taking more payday loans (payday loans come with higher interest
  than personal loans) than users of population B. Upon closer examination, however, one notes that
  only 20\% of A's population users have credit history against 55\% of B's and C's population users.
  Therefore, bussiness necessity requires that before examining statistical parity, users should be
  discriminated based on whether they have credit history or not.}
  \label{tab:S}
} \end{table*}


